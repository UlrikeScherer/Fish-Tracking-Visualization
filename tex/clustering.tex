
\documentclass[landscape]{article}
\usepackage[utf8]{inputenc}
\usepackage[T1]{fontenc}
\usepackage{graphicx}
\usepackage[a4paper, inner=1em, outer=1em, top=1em]{geometry}
\usepackage{pgffor}
\usepackage{caption}
\usepackage{subcaption}
\usepackage{hyperref}
\usepackage{ifthen}
\usepackage[nomessages]{fp}% http://ctan.org/pkg/fp
\usepackage{adjustbox}
\usepackage{tcolorbox}
\usepackage{xstring}
\usepackage{xurl}
\usepackage{etoolbox}
\usepackage{nopageno}
\usepackage{tabularx}

\providecommand{\block}{block1}
\providecommand{\tracesize}{200}
\providecommand{\nclusters}{6}
\graphicspath{{../vis/\block_trace_size_\tracesize/}} 

\title{\Large Developing exploration behavior\\
Clustering of trajectory sequences. 
{\normalsize From the 10th of September to the 8th of November.}\\
\textsc{\block}, Trace Size: \textit{\tracesize}.\\
}
\author{Research Unit 1, SCIoI Project 21}
\date{\today}


\newcommand{\subfish}[2]{%
\begin{figure}[h]%{0.7\textwidth} %height=4.2cm
    \centering
    \includegraphics[width=0.7\textwidth]{/#1.pdf}
    \caption*{Fish #2 }
\end{figure}
}

\newcommand{\fishdev}[2]{%
\begin{figure}[h]%{0.7\textwidth} %height=4.2cm
    \centering
    \foreach \i in {0,1,2,3}{
        \subweek{#1}{#2}{\i}
    }
    \caption*{Fish #2}
\end{figure}
}

\newcommand{\subweek}[3]{%
\begin{subfigure}{0.24\textwidth} %height=4.2cm
    \centering
    \includegraphics[width=\textwidth]{/#1_week\i_\tracesize.png}
    \caption*{Week #3}
\end{subfigure}
}


\begin{document}

\maketitle  
\begin{center}
\subsection*{Legend}
\begin{tabular}{p{3cm} p{8cm} }
    \textbf{Trace Size} & is \tracesize{}, thus each sample consists of 9 metrics calculated over \tracesize{} many consecutive data frames. There are 5 data frames per second.  \\
\end{tabular}
\end{center}
\newpage
%\subsection*{PCA}
\begin{figure}
    \begin{subfigure}[h]{.25\textwidth}
        \includegraphics[width=\textwidth]{/PCA_explained_variance_ratio_\tracesize.pdf}
        \subcaption*{Principal Components of 9 calculated metrics for trajectory sequence of size \tracesize{}. }
    \end{subfigure}
    \begin{subfigure}[h]{.7\textwidth}
        \includegraphics[width=\textwidth]{/PCA_loadings_\tracesize.pdf}
        \subcaption*{Loadings of the first 4 Principal Components.}
    \end{subfigure}
\end{figure}
\begin{figure}[h]
    \centering
    \begin{subfigure}[h]{.25\textwidth}
        \includegraphics[width=\textwidth]{Elbow_Method_\tracesize.pdf}
        \subcaption*{Elbow Mothod measuring decreasing distortion with increasing clusters K.}
    \end{subfigure}
    \begin{subfigure}[h]{.7\textwidth}
        \includegraphics[width=\textwidth]{cluster_characteristics_\nclusters_\tracesize.pdf}
        \subcaption*{Cluster Characteristics for \nclusters clusters}
    \end{subfigure}
\end{figure}
\begin{figure}[h]
    \centering
    \includegraphics[width=0.8\textwidth]{pca_tsne_\nclusters_\tracesize.pdf}
\end{figure}
\begin{figure}[h]
    \centering
    \includegraphics[width=0.45\textwidth]{tsne_4_\tracesize.pdf}
    \includegraphics[width=0.5\textwidth]{all_transitions_c4_\tracesize.pdf}
\end{figure}
\begin{figure}
    \centering
    \includegraphics[width=0.8\textwidth]{fish_individuality_tsne_4_\tracesize.pdf}
    \caption{Fish Individuality on four clusters. }
\end{figure}
\clearpage
\section{\nclusters{} Development Over 4 Weeks}
\foreach \fishkey in {23442333_back, 23484201_back, 23484204_back, 23520257_back, 23520258_back}% 23520264_back, 23520266_back, 23520268_back, 23520270_back, 23520276_back, 23520278_back, 23520289_back, 23442333_front, 23484201_front, 23484204_front, 23520257_front, 23520258_front, 23520264_front, 23520266_front, 23520268_front, 23520270_front, 23520276_front, 23520278_front, 23520289_front}
{
    %\subfish{fish_development/fish_development_tsne_4_\fishkey_\tracesize}{\fishkey}
    \fishdev{development/\fishkey/ft_c4_\fishkey}{\fishkey}
}




\end{document}
