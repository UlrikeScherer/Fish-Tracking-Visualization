\let\mypdfximage\pdfximage
\def\pdfximage{\immediate\mypdfximage}
\documentclass[landscape]{article}
\usepackage[utf8]{inputenc}
\usepackage[T1]{fontenc}
\usepackage{graphicx}
\usepackage[a4paper, inner=1em, outer=1em, top=1em]{geometry}
\usepackage{pgffor}
\usepackage{caption}
\usepackage{subcaption}
\usepackage{hyperref}
\usepackage{ifthen}
\usepackage[nomessages]{fp}% http://ctan.org/pkg/fp
\usepackage{adjustbox}
\usepackage{tcolorbox}
\usepackage{xstring}
\usepackage{xurl}
\usepackage{etoolbox}
\usepackage{nopageno}
\usepackage{tabularx}

\providecommand{\position}{position}
\providecommand{\block}{blockX}
\graphicspath{{../vis/\block/}} 

\title{\Large Developing exploration behavior\\
Analysis of the Trajectories over time. Visualized as sliding window plots over 10 data points for different metrics.\\
{\normalsize From the 10th of September to the 8th of November.}\\
\textsc{\block}, $\position$\\
}
\author{Research Unit 1, SCIoI Project 21}
\date{\today}

\newcommand{\subf}[1]{%
\begin{subfigure}[h]{\textwidth} %height=4.2cm
    \includegraphics[width=\textwidth]{/#1.pdf}
    %\caption*{\getsub{#1#3}, \getcsv{#1#3}}
    %\vspace{-1em}
\end{subfigure}
}


\newcommand{\plotdays}[1]{%
\begin{figure}[!hb]
    \foreach \i in {100_activity,100_tortuosity,100_turning_angle,200_entropy} {
            \subf{\i_\position#1}
            \vskip\baselineskip
    }
    \label{fig: #1}
    \caption{Metrics for days \days }
\end{figure}
}

\begin{document}

\maketitle  
\begin{center}
\subsection*{Legend}
\begin{tabular}{p{3cm} p{8cm} }
    \textbf{Activity} & is here defined as the average distance per frame (0.2 seconds) that a fish swims. We compute the mean over 100 seconds. The overall average Activity is $0.2274 cm$ per frame. \\
    \textbf{Tortuosity} & is a property of a curve being tortuous. It measures how tortuous the fish's trajectory is and scales by a distance of 10cm. For each consecutive time window where the fish swims about 10cm, we calculate the tortuosity $\tau=\frac{C}{L}$ and $\tau \in [1, \infty]$, where $C$ is the traveled distance as the cumulative sum over the frames of the time window and $L$ the direct distance from start to the end of the interval. \\
    \textbf{Turning Angle} & is the average turning angle in radians over 100 seconds. This indicates if the fish is turning more right or left at the time.\\
    \textbf{Space Use} & is computed as \textbf{entropy} for every 200 seconds. Let $p_{ij}$ be probability that the fish is in one cell of a $40\times 20$-grid, then entropy is calculated as $S=-\sum_{ij} p_{ij} \cdot ln(p_{ij})$. The entropy measures are in the scope of $e\in [0, 6.17]$. High entropy indicates randomness and four out purpose extensive space use.  \\
\end{tabular}
\begin{tabular}{llll}
    Index & Camera ID & Position & ID\\ \hline
    \input{tables/\position.tex}
\end{tabular}
    
\end{center}

\foreach \days in {00-05,06-11,12-17,18-23,24-29} {
        \plotdays{_\days}
        \vskip\baselineskip
    }

\foreach \i in {100_activity,100_tortuosity,100_turning_angle,200_entropy} {
    %\section{\StrMid{\i}{5}{100}{}}
    \newcounter{cnt}
    \csdef{m0}{activity}
    \csdef{m1}{tortuosity}
    \csdef{m2}{turning angle}
    \csdef{m3}{space use as entropy}
    \begin{figure}[!hb]
        \includegraphics[width=\textwidth]{/\i_\position.pdf}
        \caption{\block{} \csuse{m\thecnt} for all 29 days.}
        \stepcounter{cnt}
    \end{figure}
}

\end{document}
